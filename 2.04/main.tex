\documentclass[aspectratio=169, 10pt]{beamer} 
\usepackage{ctex}       % 中文支持
\usepackage{amsmath}    % 数学公式增强
\usepackage{graphicx}   % 图片支持
\usepackage{booktabs}   % 三线
\usepackage{listings}   % 代码
\usepackage{xcolor}     % 颜色支持
\AtBeginSection[]
{
  \begin{frame}
    \frametitle{本节内容} % 分隔页的标题,也可以叫 "目录" 或 "Overview"
    % currentsection: 高亮当前节
    % hideallsubsections: 隐藏所有小节(让目录更简洁,可选)
    \tableofcontents[currentsection, hideallsubsections] 
  \end{frame}
}

\usetheme{Madrid}
\usecolortheme{whale}  
\usecolortheme{orchid}  

\setbeamertemplate{navigation symbols}{}

\setmainfont{Times New Roman} 
\setsansfont{Arial}


\title[]{Agent-X 和 Fact-Audit 论文介绍} % []内为底部显示的简短标题
\subtitle{组会分享}
\author[]{LiuKai} % []内为底部显示的简短名字
\date[]{2026.2.04}


% 正文开始 
\begin{document}

% --- 封面页 ---
\begin{frame}
    \titlepage
\end{frame}

% --- 目录页 ---
\begin{frame}{目录}
    \tableofcontents
\end{frame}


\section{Agent-X 论文介绍}

\begin{frame}{Introduction}
    \begin{columns}[T, onlytextwidth]
        \begin{column}{0.48\textwidth}
            \begin{block}{主要工具}
                本篇文章的创新点除了利用 Agent 协助评价之外,还提出了从语言学的角度去评判是否是 LLM 生成的文章。大致的思路就是先用 LLM 从多个维度独立的去评判是否是机器生成文本,同时还要生成评判意见,最后由一个 Meta Agent 来综合这些维度的评判结果,给出最终的结论。这样的检测方法的优点是:
                \begin{itemize}
                    \item 无阈值,无需特定的数据集调试阈值
                    \item 可以生成详细的评判意见
                \end{itemize}
                其中这篇论文还有一个创新点是,将常用的评判指标将 AUROC 换成 Accuracy。主要理由是:论文的方法并不会输出一个模棱两可的分数,而是 Yes/No 的二分类结果,同时我们后面也会提到作者利用 Prompt 的校准工作。

            \end{block}
        \end{column}
        \begin{column}{0.48\textwidth}
            \begin{block}{Guidelines}
                \small
                \textbf{语义维度:}\\
                人类文本:做出断言的时候直接,简洁,没有大量的修饰。 \\
                机器文本: 一般在做出声明时表现得谨慎、平衡或中立。

                \textbf{文体维度:}\\
                Precision and Conciseness\\
                人类文本:简明扼要,直接引入概念。 \\
                机器文本: 平衡、解释性强且措辞谨慎\\

                \textbf{结构维度:}\\
                人类文本:句式结构与节奏多变
                机器文本:结构统一可预测且高度一致
            \end{block}
        \end{column}
    \end{columns}
\end{frame}

\begin{frame}{Methodology}
    \small
   \textbf{Agent 协作机制:} \
        论文的方法主要是分为三个 Agent 协同工作:
        \begin{itemize}
            \item Router Agent:负责利用 LLM 分析每段输入文本,联合推断其主题领域(例如:医学、法律、文学)和文体属性(例如:正式语体、论证连贯性),然后激活最相关的 Guidelines  Base Agent 进行评判,其他不相关的 Base Agent 则保持休眠。
            \item Base Agent:每个 Base Agent 都会根据 Router Agent 分发的 Guidelines, 对文本进行独立评判,输出二分类结果和评判意见。
            \item Meta Agent:负责收集所有 Base Agent 的评判结果和意见,并进行综合分析,最终输出文本是否为机器生成的结论。
        \end{itemize}
        例如一篇医学论文:Router Agent 会识别出其主题为医学,然后激活“语义清晰度"和"结构精确性"的 Base Agent 进行评判,最后 Meta Agent 会综合所有激活的 Base Agent 的评判结果,给出最终结论。
    \vspace{-0.5em}
    \begin{figure}
        \centering
        \includegraphics[width=\textwidth,height=0.4\textheight,keepaspectratio]{figures/Agentx_workflow.png}
        \caption{Agent-X Methodology}
    \end{figure}
\end{frame}

\begin{frame}{Methodology}
    关于增加精度,作者运用了 Prompt 校准的技术,避免 LLM 在判断文本中过度自信的问题。在输入给 Base Agent 的 Prompt 中,有五个对称的的词:very cautious, cautious, vanilla, confident, and very confident。
    据此, Base Agent 要输出三个内容:
    \begin{itemize}
        \item 文本是否为机器生成的二分类结果(Yes/No)
        \item 对应的置信度等级(从 very cautious 到 very confident 五个等级)
        \item 评判意见
    \end{itemize}
    其中的计算公式是:
    \begin{equation*}
        \kappa_{\mathrm{ans}} = \frac{1}{|\mathcal{P}|} \max_{y\in\{\mathrm{AI},\mathrm{Human}\}} \sum_k \mathbb{I}[f_k(x)=y]
    \end{equation*}
    \begin{equation*}
        \mu_c = \frac{1}{|\mathcal{P}|} \sum_k c_k(x),\qquad
        \sigma_c = \sqrt{\frac{1}{|\mathcal{P}|} \sum_k \bigl(c_k(x)-\mu_c\bigr)^2},\qquad
        \kappa_{\mathrm{conf}} = \frac{1}{1+\sigma_c/\mu_c}.
    \end{equation*}
    \begin{equation*}
        C_{\mathrm{cal}}(x) = \mu_c\cdot \kappa_{\mathrm{ans}} \cdot \kappa_{\mathrm{conf}}.
    \end{equation*}
    \begin{equation*}
        k^* = \arg\min_k \left| c_k(x) - C_{\mathrm{cal}}(x) \right|,\qquad f_{\mathrm{final}}(x) = f_{k^*}(x).
    \end{equation*}
     $$c_{cal} = \text{平均分} (\mu_c) \times \text{答案是否一致} (\kappa_{ans}) \times \text{置信度的一致} (\kappa_{conf})$$
\end{frame}

\begin{frame}{Methodologies}
    而在 Meta Agent 中,如果一个智能体非常自信(且经过校准验证),而另一个智能体犹豫不决,Meta Agent 会更听从自信那个智能体的意见,还会“阅读”基础智能体写的理由。如果理由写得逻辑严密、证据确凿,该智能体的意见会被优先考虑。 如果被激活的专家意见不一致,会结合语境评估并调和,分析为什么会有分歧,并试图达成一个基于共识的结论。最后生成评价意见,与 Base Agent 类似,Meta Agent 也会通过 Steering Conf 进行校准
     \begin{figure}
        \centering
        \includegraphics[width=\textwidth,height=0.65\textheight,keepaspectratio]{figures/Agentx_case.png}
        \caption{Agent-X 演示}
    \end{figure}
\end{frame}


\section{Fact-Audit 论文}



% TODO :加入对比实验
\begin{frame}{Introduction} 
    这篇论文主要是提出了一个生成用于测试 LLM 检测事实能力的数据集,这里面使用了多个 LLM 进行协作。我们先介绍它创建的数据集与其他静态数据集的区别和优劣。 \
    首先是在流程上,原来的是由人类专家给出数据,然后给 LLM 进行检测,直接评判计算准确率。在 Fact-Audit 中,使用 Agent 生成数据,这样就节省了人类成本,然后还有对 LLM 生成的检测信息进行评价,最后把结果反馈给 Agent 进行迭代优化。 \
    原有检测方法的缺点是:
    \begin{itemize}
        \item 静态数据集,如果预先训练过就会不真实
        \item 需要人工专家
        \item 只关注结果而忽略过程
    \end{itemize}
    \begin{figure}
        \centering
        \includegraphics[width=\textwidth,height=0.45\textheight,keepaspectratio]{figures/Fact_cmp.png}
        \caption{对比图}
    \end{figure}
\end{frame}

\begin{frame}{Methodology} 
    在 Methodology 上,作者解释原有的取样方法的弊端,原有的采样方法属于蒙特卡洛采样方法,在数学上十分低效,数学上的低效 ($O(1/\sqrt{N})$) 蒙特卡洛采样的收敛速度很慢。意味着你需要抽取海量的测试题($N$ 要非常大),才能准确评估出一个模型到底行不行。如果题目太少,评估结果就不可靠。同时还有个最大的缺点:大多数常见的、简单的知识。模型在训练阶段已经见得多了,随机抽样很容易抽到这些,导致评分虚高。\
    因此作者调整了采样和评分,首先定义标准期望:
    \begin{equation*}
        \mathbb{E}_{p(x)} [\mathcal{F}_\alpha(x)] = \int p(x) \mathcal{F}_\alpha(x) \mathrm{d}x
    \end{equation*}
    受重要性采样 (Importance Sampling) 的启发,引入提议分布 $q(x)$ 来提高效率,该过程调整为:
    \begin{equation*}
         \mathbb{E}_{p(x)} [\mathcal{F}_\alpha(x)] = \int q(x) \mathcal{F}_\alpha(x) \frac{p(x)}{q(x)} \mathrm{d}x = \mathbb{E}_{q(x)} \left[ \mathcal{F}_\alpha(x) \frac{p(x)}{q(x)} \right]
    \end{equation*}
    简单来说,就是引入 $q(x)$ ,来让难题有更高的采样概率,同时还调整了不同题目的分数,如果题目很难,我们可以看到,得分就会低,这样才能综合评估能力。
\end{frame}

\begin{frame}{Fact-Audit Algorithm}
    \begin{columns}[T,onlytextwidth]
        \begin{column}{0.55\textwidth}
            \begin{block}{Algorithm 1 Fact-Audit}
                \scriptsize
                \setlength{\itemsep}{0.15em}
                \begin{enumerate}
                    \item 初始化事实核查测试场景 $\Theta_0$,并设置记忆库 $\mathcal{M}=\varnothing$。
                    \item 对于 $i := 0$ 到 $n$ 重复:
                    \item \quad 将候选集 $\mathcal{X}$ 置空。
                    \item \quad \textbf{Stage 1: Prototype Emulation}
                    \item \quad 当 $|\mathcal{X}| < k$ 时:
                    \item \qquad Appraiser: $\theta_i \sim P(\Theta_i)$。
                    \item \qquad Inquirer: $x \sim q(x|\theta_i)$。
                    \item \qquad 若 $x$ 通过质量检测,则并入 $\mathcal{X}$。
                    \item \quad \textbf{Stage 2: Fact Verification with Justification}
                    \item \quad $\mathcal{M} := \mathcal{F}_\alpha(\mathcal{X}) \dfrac{p(x)}{q(x|\Theta_i)}$。
                    \item \quad 对于 $j := 0$ 到 $m$ 重复:
                    \item \qquad Prober: $x \sim \rho(\mathcal{M})$。
                    \item \qquad $\mathcal{M} := \mathcal{M} \cup \left\{ \mathcal{F}_\alpha(x) \dfrac{p(x)}{q(x|\theta_i)} \right\}$。
                    \item \quad \textbf{Stage 3: Adaptive Updating}
                    \item \quad 更新 $\Theta_{i+1} \sim \pi(\Theta_{i+1}|\Theta_i,\mathcal{M})$。
                    \item 最终返回记忆库 $\mathcal{M}$。
                \end{enumerate}
            \end{block}
        \end{column}
        \begin{column}{0.43\textwidth}
            \begin{block}{备注}
                \small
                可以看到,Appraiser 根据当前参数 $\Theta_i$ 采样一个具体的测试方向 $\theta_i$ 然后 Inquirer 根据方向 $\theta_i$ 生成具体的问题 $x$ , Quality Inspector 检查 $x$ 是否符合逻辑、是否有事实错误。只有合格的 $x$ 才会加入集合 $\mathbb{X}$ 。 

                $\mathcal{F}_{\alpha}(\mathbb{X})$ 是让被测模型回答这些问题。分式 $\frac{p(\mathbb{X})}{q(\mathbb{X}|\Theta_i)}$ 是重要性权重。

                最后 Adaptive Updating. $\Theta_{i+1} \sim \pi(\Theta_{i+1}|\Theta_i, \mathcal{M})$ 根据这一轮的测试结果,更新下一轮的生成参数 $\Theta$。
                \vspace{2.5em}
            \end{block}
        \end{column}
    \end{columns}
\end{frame}

\begin{frame}{Methodology} 
    在出题具体实现上,文章实现了三个分工明确的智能体。
    \begin{itemize}
        \item  Appraiser:主要负责出考纲,考纲分为三个经典类别,复杂主张 (Complex Claims):需要多步推理的。假新闻 (Fake News):故意误导的信息。社会谣言 (Social Rumors):社交媒体上流传的传闻。
        \item  Inquirer:根据给定的大纲出考题,考察方式也分三种:claim 模式只依靠 LLM 自身知识回答;evidence 模式引入维基百科证据检验推理;wisdom-of-crowds 模式提供模拟社交媒体评论考察群体智慧利用能力。
        \item Quality Inspector:负责“审题和校对”,确保题目质量与多样性。
    \end{itemize}
    Evaluator 则对 LLM 的回答进行评分,$\mathcal{F}_{\alpha}(x)$ 输出包含评分 $s$ 与评语 $c$。Prober 读取记忆库 $\mathcal{M}$,定位薄弱题目并重新设计试题。

    \begin{figure}
        \centering
        \includegraphics[width=\textwidth,height=0.25\textheight,keepaspectratio]{figures/Fact_framework.png}
        \caption{Fact-Audit 流水线}
    \end{figure}
\end{frame}

\begin{frame}{Methodology} 
    对于上述返回出来的结果, Appraiser 会针对记忆池 $\mathcal{M}$ 中低评分的题目进行分析,然后着重出这一方面的题目,也就是抓着 LLM 的弱点来出题,从而提高采样效率。\
    而作者在 Metric 上设置了三个指标
    \begin{itemize}
        \item  IMR: 代表低分事实核查响应占问题总数的比例。
        \item  JFR: 目标 LLM 进行了正确的判决预测但提供的理由较差的案例百分比。
        \item Grade: 分数越高,说明模型的回答越准确、理由越充分。
    \end{itemize}

\end{frame}

\begin{frame}{Limitations} 
    这篇文章的局限性主要是它只负责找问题而不负责解决问题,它可以非常精准的发现 LLM 在事实核查方面的问题,但是并没有提供解决方案。\
    未来可以考虑让框架不仅能审计模型,还能生成高质量的训练数据。这样,开发者就可以利用这些数据来微调模型,真正实现模型性能的提升,形成“评估-改进”的闭环 。
\end{frame}

\section{参考文献}
\begin{frame}[allowframebreaks]{参考文献}
    \tiny
    % 设定引用样式:
    % unsrt: [1] 排序:引用顺序 
    \bibliographystyle{unsrt}
    \nocite{*}
    \bibliography{refs}
\end{frame}

\end{document}