\documentclass[aspectratio=169, 10pt]{beamer} 
\usepackage{ctex}       % 中文支持
\usepackage{amsmath}    % 数学公式增强
\usepackage{graphicx}   % 图片支持
\usepackage{booktabs}   % 三线
\usepackage{listings}   % 代码
\usepackage{xcolor}     % 颜色支持
\AtBeginSection[]
{
  \begin{frame}
    \frametitle{本节内容} % 分隔页的标题,也可以叫 "目录" 或 "Overview"
    % currentsection: 高亮当前节
    % hideallsubsections: 隐藏所有小节(让目录更简洁,可选)
    \tableofcontents[currentsection, hideallsubsections] 
  \end{frame}
}

\usetheme{Madrid}
\usecolortheme{whale}  
\usecolortheme{orchid}  

\setbeamertemplate{navigation symbols}{}

\setmainfont{Times New Roman} 
\setsansfont{Arial}


\title[]{Agent-X 和 Fact-Audit 论文介绍} % []内为底部显示的简短标题
\subtitle{组会分享}
\author[]{LiuKai} % []内为底部显示的简短名字
\date[]{\date{\today}} 


% 正文开始 
\begin{document}

% --- 封面页 ---
\begin{frame}
    \titlepage
\end{frame}

% --- 目录页 ---
\begin{frame}{目录}
    \tableofcontents
\end{frame}


\section{Agent-X 论文介绍}

\begin{frame}{Introduction}
    \begin{columns}[T, onlytextwidth]
        \begin{column}{0.48\textwidth}
            \begin{block}{主要工具}
                本篇文章的创新点除了利用 Agent 协助评价之外,还提出了从语言学的角度去评判是否是 LLM 生成的文章。大致的思路就是先用 LLM 从多个维度独立的去评判是否是机器生成文本,同时还要生成评判意见,最后哦由一个 Meta Agent 来综合这些维度的评判结果,给出最终的结论。这样的检测方法的优点是:
                \begin{itemize}
                    \item 无阈值,无需特定的数据集调试阈值
                    \item 可以生成详细的评判意见
                \end{itemize}
                其中这篇论文还有一个创新点是,将常用的评判指标将 AUROC 换成 Accuracy。主要理由是:论文的方法并不会输出一个模棱两可的分数,而是 Yes/No 的二分类结果,同时我们后面也会提到作者利用 Prompt 的校准工作。

            \end{block}
        \end{column}
        \begin{column}{0.48\textwidth}
            \begin{block}{Guidelines}
                \small
                \textbf{语义维度:}\\
                人类文本:做出断言的时候直接,简洁,没有大量的修饰。 \\
                机器文本: 一般在做出声明时表现得谨慎、平衡或中立。

                \textbf{文体维度:}\\
                Precision and Conciseness\\
                人类文本:简明扼要,直接引入概念。 \\
                机器文本: 平衡、解释性强且措辞谨慎\\

                \textbf{结构维度:}\\
                人类文本:句式结构与节奏多变
                机器文本:结构统一可预测且高度一致
            \end{block}
        \end{column}
    \end{columns}
\end{frame}

\begin{frame}{Methodology}
    \small
   \textbf{Agent 协作机制:} \
        论文的方法主要是分为三个 Agent 协同工作:
        \begin{itemize}
            \item Router Agent:负责利用 LLM 分析每段输入文本,联合推断其主题领域(例如:医学、法律、文学)和文体属性(例如:正式语体、论证连贯性),然后激活最相关的 Guidelines  Base Agent 进行评判,其他不相关的 Base Agent 则保持休眠。
            \item Base Agent:每个 Base Agent 都会根据 Router Agent 分发的 Guidelines, 对文本进行独立评判,输出二分类结果和评判意见。
            \item Meta Agent:负责收集所有 Base Agent 的评判结果和意见,并进行综合分析,最终输出文本是否为机器生成的结论。
        \end{itemize}
        例如一篇医学论文:Router Agent 会识别出其主题为医学,然后激活“语义清晰度"和"结构精确性"的 Base Agent 进行评判,最后 Meta Agent 会综合所有激活的 Base Agent 的评判结果,给出最终结论。
    \vspace{-0.5em}
    \begin{figure}
        \centering
        \includegraphics[width=\textwidth,height=0.4\textheight,keepaspectratio]{figures/Agentx_workflow.png}
        \caption{Agent-X Methodology}
    \end{figure}
\end{frame}

\begin{frame}{Methodology}
    关于增加精度,作者运用了 Prompt 校准的技术,避免 LLM 在判断文本中过度自信的问题。在输入给 Base Agent 的 Prompt 中,有五个对称的的词:very cautious, cautious, vanilla, confident, and very confident。
    据此, Base Agent 要输出三个内容:
    \begin{itemize}
        \item 文本是否为机器生成的二分类结果(Yes/No)
        \item 对应的置信度等级(从 very cautious 到 very confident 五个等级)
        \item 评判意见
    \end{itemize}
    其中的计算公式是:
    \begin{equation*}
        \kappa_{\mathrm{ans}} = \frac{1}{|\mathcal{P}|} \max_{y\in\{\mathrm{AI},\mathrm{Human}\}} \sum_k \mathbb{I}[f_k(x)=y]
    \end{equation*}
    \begin{equation*}
        \mu_c = \frac{1}{|\mathcal{P}|} \sum_k c_k(x),\qquad
        \sigma_c = \sqrt{\frac{1}{|\mathcal{P}|} \sum_k \bigl(c_k(x)-\mu_c\bigr)^2},\qquad
        \kappa_{\mathrm{conf}} = \frac{1}{1+\sigma_c/\mu_c}.
    \end{equation*}
    \begin{equation*}
        C_{\mathrm{cal}}(x) = \mu_c\cdot \kappa_{\mathrm{ans}} \cdot \kappa_{\mathrm{conf}}.
    \end{equation*}
    \begin{equation*}
        k^* = \arg\min_k \left| c_k(x) - C_{\mathrm{cal}}(x) \right|,\qquad f_{\mathrm{final}}(x) = f_{k^*}(x).
    \end{equation*}
     $$c_{cal} = \text{平均分} (\mu_c) \times \text{答案是否一致} (\kappa_{ans}) \times \text{置信度的一致} (\kappa_{conf})$$
\end{frame}

\begin{frame}{Methodologies}
    而在 Meta Agent 中,如果一个智能体非常自信(且经过校准验证),而另一个智能体犹豫不决,Meta Agent 会更听从自信那个智能体的意见,还会“阅读”基础智能体写的理由。如果理由写得逻辑严密、证据确凿,该智能体的意见会被优先考虑。 如果被激活的专家意见不一致,会结合语境评估并调和,分析为什么会有分歧,并试图达成一个基于共识的结论。最后生成评价意见,与 Base Agent 类似,Meta Agent 也会通过 Steering Conf 进行校准
     \begin{figure}
        \centering
        \includegraphics[width=\textwidth,height=0.65\textheight,keepaspectratio]{figures/Agentx_case.png}
        \caption{Agent-X 演示}
    \end{figure}
\end{frame}


\section{Fact-Audit 论文}



\section{参考文献}
\begin{frame}[allowframebreaks]{参考文献}
    \tiny
    % 设定引用样式:
    % unsrt: [1] 排序:引用顺序 
    \bibliographystyle{unsrt}
    \nocite{*}
    \bibliography{refs}
\end{frame}




\end{document}